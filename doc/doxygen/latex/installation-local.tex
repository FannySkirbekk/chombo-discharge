\subsection*{H\+D\+F5 installation }

Before attempting to install H\+D\+F5, please make sure that C++, Fortran, and M\+PI compilers are in place.

H\+D\+F5 (Hierarchical Data Format) is a portable, binary data file with parallel support and is used to reduce I/O bottlenecks in simulations. H\+D\+F5 can be obtained from \href{https://support.hdfgroup.org/HDF5/}{\tt here}.

While it is possible to compile our code for serial execution only, we generally recommend that the user installs both serial and parallel versions of H\+D\+F5. The two versions should {\itshape not} be installed in the same directory. For versions of H\+D\+F5 newer than 1.\+6, we require that the user configures H\+D\+F5 with the 1.\+6 A\+PI version.


\begin{DoxyEnumerate}
\item Compile and install zlib (a compression library). zlib can be obtained through \begin{DoxyVerb} sudo apt-get install zlib1g-dev
\end{DoxyVerb}

\item Download and install H\+D\+F5 for {\itshape serial execution}. \begin{DoxyVerb} ./configure --prefix=/usr/local/hdf5 --enable-production --enable-cxx --enable-fortran --with-default-api-version=v16
 make
 make install
\end{DoxyVerb}


This will install a serial version of H\+D\+F5 in /usr/local/hdf5 (if you want to install
\item Install H\+D\+F5 for {\itshape parallel execution} \begin{DoxyVerb} ./configure --prefix=/usr/local/phdf5 --enable-production --enable-fortran --with-default-api-version=v16 --enable-parallel
 make
 make install
\end{DoxyVerb}


This will install a parallel version of H\+D\+F5 in /usr/local/phdf5
\end{DoxyEnumerate}

\subsection*{Building the source code }

Once sequential and parallel versions of H\+D\+F5 are installed, the source code may be built. The source is built by inheriting a set of makefiles from Chombo. By default, a version of Chombo-\/3.\+2 is shipped with this code. To build the source code, you must ensure that the makefile finds the Chombo library by modifying the G\+N\+Umakefile which lies at the top directory in the code. Modify this file so that {\ttfamily C\+H\+O\+M\+B\+O\+\_\+\+H\+O\+ME} points to the correct location (a full path specification may be necessary). For example, if the code is in the home folder of a user {\ttfamily example\+User}, {\ttfamily C\+H\+O\+M\+B\+O\+\_\+\+H\+O\+ME} should be \begin{DoxyVerb}CHOMBO_HOME := /home/exampleUser/FluidAMRStreamer/Chombo-3.2/lib
\end{DoxyVerb}


Specification of default settings for Chombo are necessary for compiling both the Chombo library and the source code. You should modify Make.\+defs.\+local which lies in Chombo-\/3.\+2/lib/mk/ for specification of compilers and path to the serial and parallel H\+D\+F5 installations. If you installed H\+D\+F5 in /usr/local/hdf5 and /usr/local/phdf5, the default settings in Make.\+defs.\+local are (probably) sufficient.

The streamer code can be compiled for two-\/dimensional and three-\/dimensional executation on Cartesian grids (default is 2D). To compile the code, it should be sufficient to navigate to one of the plasma models and then execute \begin{DoxyVerb}make "model"
\end{DoxyVerb}


where \char`\"{}model\char`\"{} is replaced by the model name. If you want to silence G\+N\+Umake during the compilation stage, include a flag -\/s. I.\+e. \begin{DoxyVerb}make -s "model"
\end{DoxyVerb}


This will inherit a set of makefiles\+: Chombo source codes are compiled first and the streamer code is compiled second. To execute for three-\/dimensional execution, either make appropriate changes to Make.\+defs.\+local (D\+IM=3), or run \begin{DoxyVerb}make DIM=3 "model"
\end{DoxyVerb}


By default, the executable will be named according to the specifications in Make.\+defs.\+local. For example, if the code was compiled for optimized three-\/dimensional execution using M\+PI (mpi\+CC as compiler) gfortran on a Linux system, the executable is named \begin{DoxyVerb}plasma3d.Linux.64.mpiCC.gfortran.OPT.MPI.ex
\end{DoxyVerb}


The code has compiled successfully if this file is created. 