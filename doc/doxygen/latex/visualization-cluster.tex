To use Vis\+It\textquotesingle{}s parallel computing capabilities to its fullest, its best to run it in server mode. To do this, {\itshape you should have matching versions of Vis\+It both locally and remotely}.

To set up remote visualization, you will need


\begin{DoxyEnumerate}
\item The host address.
\item The path to Vis\+It on the server
\item Optionally have to modify your .bashrc (or equivalent) so that Vis\+It is loaded on ssh login.
\end{DoxyEnumerate}

Below, we show how to set up a host profile for fram.\+sigma2.\+no

\subsection*{Remote visualization on fram.\+sigma2.\+no }

We will use the remote 2.\+13.\+0 remote installation of Vis\+It on fram. Firstly, modify your .bashrc file so that your login automatically loads the necessary modules. On fram, this is done by appending the following line to your .bashrc file \begin{DoxyVerb}  module load VisIt/2.13.0-intel-2017a
\end{DoxyVerb}


Next, we will set up the host profile on your local installation. Run Vis\+It2.\+13.\+0 locally


\begin{DoxyEnumerate}
\item Go to Options-\/$>$Host profiles
\item Create a new host profile
\item In the field \textquotesingle{}Host name\textquotesingle{}, use any host name.
\item In the field \textquotesingle{}Remote host name\textquotesingle{}, type \textquotesingle{}fram.\+sigma2.\+no\textquotesingle{}
\item In the field \textquotesingle{}Path to Vis\+It installation\textquotesingle{}, type \textquotesingle{}/cluster/software/\+Vis\+It/2.13.\+0-\/intel-\/2017a\textquotesingle{}
\item Check the \textquotesingle{}Tunnel data connections through S\+SH\textquotesingle{} flag.
\item Navigate to the \textquotesingle{}Launch profiles\textquotesingle{} tab.
\item Create a new profile and navigate to the \textquotesingle{}Parallel\textquotesingle{} tab.
\item In the \textquotesingle{}Parallel\textquotesingle{} tab, check the \textquotesingle{}Launch parallel engine\textquotesingle{} box.
\end{DoxyEnumerate}

To open a file remotely. Select \textquotesingle{}Open\textquotesingle{} and choose your new host under the \textquotesingle{}Host\textquotesingle{} dropdown menu. You will be prompted for a password, after which you can navigate to your remote file. 